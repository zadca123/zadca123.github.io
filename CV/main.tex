%%%%%%%%%%%%%%%%%%%%%%%%%%%%%%%%%%%%%%%%%
% Developer CV
% LaTeX Template
% Version 1.0 (28/1/19)
%
% This template originates from:
% http://www.LaTeXTemplates.com
%
% Authors:
% Jan Vorisek (jan@vorisek.me)
% Based on a template by Jan Küster (info@jankuester.com)
% Modified for LaTeX Templates by Vel (vel@LaTeXTemplates.com)
%
% License:
% The MIT License (see included LICENSE file)
%
%%%%%%%%%%%%%%%%%%%%%%%%%%%%%%%%%%%%%%%%%

%----------------------------------------------------------------------------------------
%	PACKAGES AND OTHER DOCUMENT CONFIGURATIONS
%----------------------------------------------------------------------------------------

\documentclass[9pt]{developercv} % Default font size, values from 8-12pt are recommended

%----------------------------------------------------------------------------------------

\begin{document}

%----------------------------------------------------------------------------------------
%	TITLE AND CONTACT INFORMATION
%----------------------------------------------------------------------------------------

\begin{minipage}[t]{0.45\textwidth} % 45% of the page width for name
    \vspace{-\baselineskip} % Required for vertically aligning minipages

    % If your name is very short, use just one of the lines below
    % If your name is very long, reduce the font size or make the minipage wider and reduce the others proportionately
    \colorbox{black}{{\HUGE\textcolor{white}{\textbf{\MakeUppercase{Adam}}}}} % First name

    \colorbox{black}{{\HUGE\textcolor{white}{\textbf{\MakeUppercase{Salwowski}}}}} % Last name

    \vspace{6pt}

    {\huge Kandydat na stażystę} % Career or current job title
\end{minipage}
\begin{minipage}[t]{0.3\textwidth} % 27.5% of the page width for the first row of icons
    \vspace{-\baselineskip} % Required for vertically aligning minipages

    % The first parameter is the FontAwesome icon name, the second is the box size and the third is the text
    % Other icons can be found by referring to fontawesome.pdf (supplied with the template) and using the word after \fa in the command for the icon you want
    \icon{MapMarker}{12}{Tęczowa 17/24, Olsztyn}\\
    \icon{Phone}{12}{+48 572 313 319}\\
    \icon{At}{12}{\href{mailto:150998@student.uwm.edu.pl}{150998@student.uwm.edu.pl}}\\
\end{minipage}
\begin{minipage}[t]{0.275\textwidth} % 27.5% of the page width for the second row of icons
    \vspace{-\baselineskip} % Required for vertically aligning minipages

    % The first parameter is the FontAwesome icon name, the second is the box size and the third is the text
    % Other icons can be found by referring to fontawesome.pdf (supplied with the template) and using the word after \fa in the command for the icon you want
    % \icon{Globe}{12}{\href{https://alyx.vance.me}{alyx.vance.me}}\\
    \icon{Github}{12}{\href{https://github.com/zadca123}{github.com/zadca123}}\\
    \icon{Gitlab}{12}{\href{https://gitlab.com/zadca123}{gitlab.com/zadca123}}\\
\end{minipage}

\vspace{0.5cm}

%----------------------------------------------------------------------------------------
%	INTRODUCTION, SKILLS AND TECHNOLOGIES
%----------------------------------------------------------------------------------------

\cvsect{Kim jestem?}

\begin{minipage}[t]{0.4\textwidth} % 40% of the page width for the introduction text
    \vspace{-\baselineskip} % Required for vertically aligning minipages

    Student \textbf{Uniwersytetu Warmińsko-Mazurskiego} na \textbf{Wydziale Matematyki i Informatyki}, kierunk: \textbf{Inżynieria Systemów Informatycznych}. Pasjonat technologii, prywatności w sieci.\\ % Dummy text
\end{minipage}
\hfill % Whitespace between
\begin{minipage}[t]{0.5\textwidth} % 50% of the page for the skills bar chart
    \vspace{-\baselineskip} % Required for vertically aligning minipages
    \begin{barchart}{5.5}
        \baritem{Python}{70}
        \baritem{Bash/Shell Script}{70}
        \baritem{Linux SysAdmin}{80}
        \baritem{Git}{40}
        \baritem{LaTeX}{30}
        \baritem{Markdown/Org-mode}{60}
    \end{barchart}
\end{minipage}

\begin{center}
    \bubbles{4/Emacs, 5/vim, 4/git, 2/Office, 2/Gimp, 4/Regex}
\end{center}

%----------------------------------------------------------------------------------------
%	EXPERIENCE
%----------------------------------------------------------------------------------------

\cvsect{Doświadczenie}

\begin{entrylist}
    \entry
    {2018 --- obecnie\\\footnotesize{full time}}
    {Student}
    {Wmii UWM}
    {Na studiach miałem styczność z wieloma językami programowania czy algorytmami. Obecnie mamy zajęcia związane z Data Mining\\ \texttt{HTML/CSS/JS}\slashsep\texttt{PHP}\slashsep\texttt{C/C++}\slashsep\texttt{Python}\slashsep\texttt{Java}\slashsep\texttt{Haskell}\slashsep\texttt{Perl}\slashsep\texttt{UML}}
    \entry
    {2019 --- obecnie\\\footnotesize{part time}}
    {Hobbystyczne programowanie}
    {}
    {Piszę sktrypty i programy automatyzujące żmudne procesy lub wrappery, które ułatwiają wchodzenie w interakcję ze skomplikowamymi narzędziami. Pracowałem także w grupie i wdrażałem aplikację na chmurę, czy hosting.\\ \texttt{Python}\slashsep\texttt{Bash/Shell}\slashsep\texttt{Emacs-Lisp}\slashsep\texttt{Django}\slashsep\texttt{Docker}\slashsep\texttt{Heroku}\slashsep\texttt{Azure}}
    \entry
    {2017 --- obecnie\\\footnotesize{full time}}
    {Użytkownik Linuxa}
    {}
    {Już od 2017 zacząłem mieć styczność z tym systemem operacyjnym i wieloma jego dystrybucjami, a w 2019 całkowicie się na niego przerzuciłem. Jestem także biegły w posługiwaniu się edytorami tekstowym vim czy Emacs, używaniu i łączeniu różnych narzędzi z pakietu GNU, obróbka audio/video za pomocą ffmpeg czy imagemagick.\\ \texttt{Linux}\slashsep\texttt{GNU}\slashsep\texttt{UNIX}}
\end{entrylist}

%----------------------------------------------------------------------------------------
%	EDUCATION
%----------------------------------------------------------------------------------------

\cvsect{Wykształcenie}

\begin{entrylist}
    % \entry
    % {2013 --- 2017}
    % {Wmii UWM}
    % {}
    % {\lorem\lorem\lorem}
    \entry
    {2013 --- 2017}
    {Technikum Mechaniczne w Kolnie}
    {}
    {W tej szkole zdobyłem wiedzę na temat mechaniki oraz obsługi maszyn skrawających i CNC.}
    % \entry
    % {1998 --- obecnie\\\footnotesize{part time}}
    % {Rolnik}
    % {}
    % {Od wczesnych lat uczyłem się o rolnictwie, uprawie ziemi, czy hodowli bydła.}
\end{entrylist}

%----------------------------------------------------------------------------------------
%	ADDITIONAL INFORMATION
%----------------------------------------------------------------------------------------

\begin{minipage}[t]{0.3\textwidth}
    \vspace{-\baselineskip} % Required for vertically aligning minipages
    \cvsect{Języki}

    \textbf{Polish} --- ojczysty\\
    \textbf{English} --- biegły\\
    \textbf{German} --- początkujący\\
    \textbf{Russian} --- początkujący
\end{minipage}
\hfill
\begin{minipage}[t]{0.3\textwidth}
    \vspace{-\baselineskip} % Required for vertically aligning minipages
    \cvsect{Zainteresowania}

    Lubię oglądać i czytać o historii, uprawiać sport, majstrować przy komputerze i pisać skrypty, które automatyzują różne rzeczy, testować różne oprogramowania i szukać alternatyw.
\end{minipage}
\hfill
\begin{minipage}[t]{0.3\textwidth}
	\vspace{-\baselineskip} % Required for vertically aligning minipages

	\cvsect{Przyszłość}

    Mam zamiar za pośrednictwem pythona przyjrzeć się bardziej Machine Learning, Data Mining czy frameworkowi \textbf{FastAPI}. Zainteresowały mnie także języki programowania takie jak \textbf{Rust} czy \textbf{Lisp} i jego dialekty.
\end{minipage}

%----------------------------------------------------------------------------------------

\end{document}
